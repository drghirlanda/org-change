% Created 2023-05-04 Thu 12:10
% Intended LaTeX compiler: pdflatex
\documentclass[11pt]{article}
\usepackage[utf8]{inputenc}
\usepackage[T1]{fontenc}
\usepackage{graphicx}
\usepackage{longtable}
\usepackage{wrapfig}
\usepackage{rotating}
\usepackage[normalem]{ulem}
\usepackage{amsmath}
\usepackage{amssymb}
\usepackage{capt-of}
\usepackage{hyperref}
\hypersetup{hidelinks}
\author{Stefano Ghirlanda}
\date{\today}
\title{org-change: Annotate changes in org-mode}
\usepackage{changes}
\begin{document}

\maketitle

\section{Introduction}
\label{sec:org598d7d6}

org-change is an Emacs minor mode that (ab)uses the org-mode link
syntax for a simple ``track changes'' feature similar to some word
processors. The main use case is for authors to highlight the changes
between two versions of a document, such as before and after review by
a third party. For this, org-change provides:
\begin{itemize}
\item A \texttt{change} link type to mark additions, deletions, and replacements.
\item Functions and key bindings to manipulate \texttt{change} links.
\item Export filters for \texttt{change} links (currently Latex and HTML).
\end{itemize}

\section{Installation}
\label{sec:orgb3bc902}

Install from MELPA, or manually from here.

\section{\texttt{change} link syntax}
\label{sec:org21fb566}

To indicate that ``old text'' is being replaced by ``new text,''
org-change defines the following \texttt{change} link syntax:
\begin{verbatim}
[[change:old text][new text]]
\end{verbatim}
The idea is that you end up seeing only ``new text,'' because org-mode
(typically) hides the part of the link within the first pair of
brackets. To indicate an addition, org-change just omits \texttt{old text}:
\begin{verbatim}
[[change:][new text]]
\end{verbatim}
To indicate a deletion, org-change uses \texttt{((DELETED))} as \texttt{new text}:
\begin{verbatim}
[[change:old text][((DELETED))]]
\end{verbatim}
You can embed comments in change links by surrounding them with double
stars at the end of \texttt{new text}:
\begin{verbatim}
[[change:old text][new text**A comment**]]
\end{verbatim}
All this by itself is not very useful, but read on.

\section{\texttt{change} link manipulation}
\label{sec:org364c7f1}

org-change provides key sequences to easily manipulate \texttt{change}
links. All key sequences start with \texttt{C-`} (control + left quote). Not
the prettiest, but few control prefixes are free. It's the curse of
keymensionality. To change it, see section \ref{sec:org0e35391}.

The key sequences are:
\begin{description}
\item[{\texttt{C-` a}}] for additions.
\item[{\texttt{C-` d}}] for deletions.
\item[{\texttt{C-` r}}] for replacements.
\end{description}
All these act on the active region. For example, in the case of
replacement, the region is marked for deletion, and you are prompted
for new text. You can also use \texttt{C-` a} without marking a region, in
which case you are prompted for new text.

Key sequences are also provided to accept or reject the change under
the cursor:
\begin{description}
\item[{\texttt{C-` k}}] to accept.
\item[{\texttt{C-` x}}] to reject.
\end{description}
``Accept'' means to delete the change link (including any comments) and
insert the new text (or nothing, if the change is a
deletion). ``Reject'' means to delete the change link and insert the old
text (or nothing, if the change is an addition). 

If there is no change under the cursor, accept and reject work on all
change links in the active region. If there is no active region,
nothing happens. You can accept or reject all changes in a document by
selecting the whole buffer, but note that this deletes all changes. If
you just want to export a clean manuscript, see section \ref{sec:org709676f}.

The following functionality is provided by org-mode, and is useful for
\texttt{change} links:
\begin{description}
\item[{\texttt{C-c C-l}}] lets you edit the link in the minibuffer. Because this
is an org-mode function for all links, it will display the ``old
text'' as \texttt{Link: change:old text} and the ``new text'' as \texttt{Description: new text}.
\item[{\texttt{M-x org-toggle-link-display}}] toggles between showing and hiding
the hidden part of every link in the buffer. This can be useful to
work on longer edits.
\end{description}

\section{Exporting}
\label{sec:orgc8f0e92}
\subsection{\LaTeX{} export}
\label{sec:orga5fb52b}

When exporting to \LaTeX{}, org-change uses the \texttt{changes} package, which
it includes automatically in the exported document. org-change will
then use the commands \texttt{\textbackslash{}added}, \texttt{\textbackslash{}deleted}, and \texttt{\textbackslash{}replaced} provided
by this package.

org-change supports some additional features of the \texttt{changes}
package. It supports comments, so that
\begin{verbatim}
[[change:old text][new text**A comment**]]
\end{verbatim}
is exported to
\begin{verbatim}
\replaced[comment=A comment]{new text}{old text}
\end{verbatim}
You can also sneak in other fields supported by \texttt{changes} at the end
of the comment. For example, you can indicate the author of the
comment:
\begin{verbatim}
[[change:old text][new text**My comment,author=SG**]]
\end{verbatim}
which is exported to:
\begin{verbatim}
\replaced[comment=My comment,author=SG]{new text}{old text}
\end{verbatim}
Lastly, you can set options for the \texttt{changes} package by setting the
variable \texttt{org-change-latex-options}. For example, you can place this
code somewhere in your document and evaluate it:
\begin{verbatim}
#+begin_src elisp
  (setq org-change-latex-options "[markup=underline]")
#+end_src
\end{verbatim}
Note that you need to include the brackets. The \texttt{changes} package also
has configurations that are not set through package options, which you
can set through \texttt{\#+latex\_header:} lines.

The \texttt{changes} package causes errors with some \LaTeX{} commands. This can
happen, for example, when \texttt{\textbackslash{}cite} and similar commands appear in a
change. To fix these problems, you can try to add \texttt{\textbackslash{}protect} or
\texttt{\textbackslash{}noexpand} before the offending command, or to wrap the command in an
\texttt{\textbackslash{}mbox}.

\subsection{HTML export}
\label{sec:org98ae360}

When exporting to HTML, org-change produces \texttt{<span>} elements with
classes \texttt{org-change-added}, \texttt{org-change-deleted}, and
\texttt{org-change-comment}. A replace link has both an added and a deleted
span, while add and delete links only have one span. The comment span
is embedded in the add span when present, otherwise in the delete
span. So this:
\begin{verbatim}
[[change:old text][new-text**comment**]]
\end{verbatim}
becomes this:
\begin{verbatim}
<span class="org-change-added">
  new text
  <span class="org-change-comment">
    comment
  </span>
</span>
<span class="org-change-deleted">
  old text
</span>
\end{verbatim}
You can then use CSS to display these classes as desired.

\subsection{Producing a clean document}
\label{sec:org709676f}

When exporting, org-change looks first at the variable
\texttt{org-change-final}. This is initially \texttt{nil}, meaning that the export
proceeds according to the selected backend as detailed above. If
\texttt{org-change-final} is not \texttt{nil}, then only the new text is exported,
resulting in a ``clean'' document without change markup. To achieve
this, you can evaluate this code block before exporting:
\begin{verbatim}
#+begin_src elisp :exports none :results silent
  (setq org-change-final t)
#+end_src
\end{verbatim}
This code can be anywhere in your file, even a \texttt{:noexport:} section.

\section{Customizing and extending}
\label{sec:org0e35391}
\subsection{Customization}
\label{sec:org7e54231}

The key sequences and the face used to display change links can be
changed through the customize interface:
\begin{verbatim}
M-x customize-group RET org-change
\end{verbatim}

\subsection{Adding exporters}
\label{sec:orga0d0fa0}

To add an export format, add something like this to your org file:
\begin{verbatim}
#+begin_src elisp
  (org-change-add-export-backend 'backend 'backend-function)
#+end_src
\end{verbatim}
where \texttt{backend} is a backend known to org-mode and \texttt{backend-function}
is a function that produces the desired string from three string
arguments: \texttt{old-text}, \texttt{new-text}, and \texttt{comment}. The function can
figure out whether the change is an addition, deletion, or replacement
by looking at these variables: for additions, \texttt{old-text} is empty; for
deletions, \texttt{new-text} is \texttt{((DELETED))}; other cases are replacements.

\section{Bugs and limitations}
\label{sec:org5dd94b3}

Please submit bugs as \href{https://github.com/drghirlanda/org-change/issues}{issues on Github}.

\begin{itemize}
\item The content of the change link can contain org-mode notation like
\textbf{bold} and \emph{emphasis}, as well as Latex code. However, some other
features do not currently work. Notably, org-ref links must be
translated manually to Latex. So this will \textbf{not} work:
\begin{verbatim}
[[change:][Let's cite something cite:&something1972]]
\end{verbatim}
But this will:
\begin{verbatim}
[[change:][Let's cite something \cite{something1972}]]
\end{verbatim}
\item Link hiding is sometimes inaccurate in org-mode. You may see stray
brackets especially with link that span multiple lines. Sometimes \texttt{M-q}
takes care of this, or you can enable \texttt{visual-line-mode} and keep
paragraphs as single unbroken lines.
\item \LaTeX{} export is not fully compatible with HTML export if you use the
extended comment syntax. That is, HTML export does not handle extra
arguments like ``author=SG,'' which are a feature of the \texttt{changes}
package for \LaTeX{}.
\end{itemize}

\section{Planned features}
\label{sec:org3f02b67}

Please submit feature requests as \href{https://github.com/drghirlanda/org-change/issues}{issues on Github}.

\begin{itemize}
\item Simple HTML CSS for change markup.
\item More export filters?
\end{itemize}

\section{Notes}
\label{sec:orgc26538c}

To get started on org-change, I described some features to ChatGPT
(April 2023 version) and asked for the corresponding code. It was
wrong in many ways, like using non-existing functions with plausible
names (\texttt{org-escape-latex}) and other non-existing features. It also
insisted that some things would work even when told that they did
not. It did have a good grasp of many things, like defining a minor
mode and customize variables, and it was always syntactically correct.
\end{document}